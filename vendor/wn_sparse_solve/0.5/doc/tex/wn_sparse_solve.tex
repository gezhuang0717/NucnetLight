%///////////////////////////////////////////////////////////////////////////////
% <file type="public">
%
% <license>
%   See the src/README.txt file for this module for copyright and license
%   information.
% </license>
%   <description>
%     <abstract>
%       This is the Webnucleo Report describing the computational details
%       behind the wn_sparse_solve module.
%     </abstract>
%     <keywords>
%       wn_sparse_solve, report
%     </keywords>
%   </description>
%
%   <authors>
%     <current>
%       <author userid="mbradle" start_date="2008/06/08" />
%     </current>
%     <previous>
%     </previous>
%   </authors>
%
%   <compatibility>
%     TeX (Web2C 7.4.5) 3.14159 kpathsea version 3.4.5
%   </compatibility>
%
% </file>
%///////////////////////////////////////////////////////////////////////////////
%
% This is a sample LaTeX input file.  (Version of 9 April 1986)
%
% A '%' character causes TeX to ignore all remaining text on the line,
% and is used for comments like this one.

\documentclass{article}    % Specifies the document style.

\usepackage{hyperref}

                           % The preamble begins here.
\title{Webnucleo Technical Report: wn\_sparse\_solve Module}  

\author{Bradley S. Meyer}
%\date{December 12, 1984}   % Deleting this command produces today's date.

\begin{document}           % End of preamble and beginning of text.

\maketitle                 % Produces the title.

wn\_sparse\_solve is a Webnucleo module that provides a convenient interface
between Webnucleo matrix codes and SPARSKIT2, a toolkit for basic sparse
matrix operations by Yousef Saad.  This technical report provides information
about the wn\_sparse\_solve structures, about the way users can provide
their own preconditioner routines for solutions to sparse matrix equations,
and about stopping criteria.
The latest release of wn\_sparse\_solve is available at
\begin{center}
http://www.webnucleo.org/home/modules/wn\_sparse\_solve/
\end{center}

\section{wn\_sparse\_solve Structures Overview}  

wn\_sparse\_solve has three basic structures.  These structures contain the
settings for sparse matrix solutions, and the user can update the settings
through wn\_sparse\_solve API routines.  The structures are the following:

\begin{itemize}

\item WnSparseSolve\_\_Mat

This structure contains the settings for the SPARSKIT2 basic sparse solver
routines, which solve the matrix equation

\begin{equation}
A x = b
\label{eq:axb}
\end{equation}

where $A$ is a matrix, $b$ is the known right-hand-side (rhs) vector, and
$x$ is the unknown vector.  SPARSKIT2 has a variety of iterative solvers
available to find solutions to Eq. (\ref{eq:axb}), and WnSparseSolve\_\_Mat
permits Webnucleo module users a straightforward interface to them.

The user creates a solver with the API routine
{\em WnSparseSolve\_\_Mat\_\_new()}.  The default settings for the solver are
1) the solver method is the biconjugate gradient method, 2) the maximum number
of iterations the solver will use is 100, 3) the relative tolerance is
$10^{-8}$, 4) the absolute tolerance is $10^{-8}$, 5) there is no
preconditioning of the matrix, 6) the stopping criteria are relative to
the initial residual, and 7) debugging is turned off.  The user
may then update these settings through API routines.

\item WnSparseSolve\_\_Exp

This structure contains the settings for use with SPARSKIT2's routines for
solving the matrix equation

\begin{equation}
\frac{dY}{dt} = AY
\label{eq:exp_dydt}
\end{equation}

where $Y$ is a vector, $A$ is a matrix, and $t$ is the time.  In general
one seeks the solution $Y(t + \Delta t)$ given $Y(t)$.  The formal
solution is by exponentiation of the matrix:

\begin{equation}
Y( t + \Delta t ) = exp(A \Delta t) Y( t )
\label{eq:exp_sol}
\end{equation}

To find the solution, wn\_sparse\_solve uses a modified form of the
SPARSKIT2 unsupported routine exppro.f.  The version of exppro.f used
is in the src/ directory in the wn\_sparse\_solve installation.

The user creates a new exponentiation solver with the API routine
{\em WnSparseSolve\_\_Exp\_\_new()}.  The default settings for the solver
are 1) the maximum number of iterations is 100, 2) the tolerance is 
$10^{-8}$, 3) the workspace setting is 40 (the workspace setting must be 
in the range between 1 and 60, inclusive), and 4) debugging is turned off.
The user can update these settings through API routines.

\item WnSparseSolve\_\_Phi

This structure contains the settings for use with SPARSKIT2's routines for
solving the matrix equation

\begin{equation}
\frac{dY}{dt} = AY + P
\label{eq:phi_dydt}
\end{equation}

where $Y$ is a vector, $A$ is a matrix, $P$ is a constant matrix,
and $t$ is the time.  In general
one seeks the solution $Y(t + \Delta t)$ given $Y(t)$.

To find the solution, wn\_sparse\_solve uses a modified form of the
SPARSKIT2 unsupported routine phipro.f.  The version of phipro.f used
is in the src/ directory in the wn\_sparse\_solve installation.

The user creates a new exponentiation plus constant vector
solver with the API routine
{\em WnSparseSolve\_\_Phi\_\_new()}.  The default settings for the solver
are 1) the maximum number of iterations is 100, 2) the tolerance is 
$10^{-8}$, 3) the workspace setting is 40 (the workspace setting must be 
in the range between 1 and 60, inclusive), and 4) debugging is turned off.
The user can update these settings through API routines.

\end{itemize}

\section{Preconditioning}

The rate of convergence of an iterative matrix equation solver typically
depends on the condition number of the matrix.  A larger condition number
will generally lead to a slower convergence.  Suppose one is solving the matrix
equation in Eq. (\ref{eq:axb}).  A preconditioner is a matrix $P$ whose inverse
is approximately the inverse of $A$.  Applying $P^{-1}$ to $A$ can reduce
the condition number and lead to a more rapid convergence of the solution.

A left preconditioner can be applied to the original equation to give
\begin{equation}
P^{-1} A x = P^{-1} b.
\label{eq:pre_left}
\end{equation}
Since $P^{-1} \approx A^{-1}$, $P^{-1} b$ is an approximation to $x$.  For
a right preconditioner, one writes
\begin{equation}
AP^{-1}P x = b.
\label{eq:pre_right}
\end{equation}
One again finds $x \approx P^{-1} b$ for $P^{-1} \approx A^{-1}$.

wn\_sparse\_solve allows a user to supply his or her own left preconditioner.
In particular, a user supplies two routines.  One takes in a vector $y$ and
returns $x$, where $x$ is the result of solving
\begin{equation}
P x = y.
\end{equation}
$P$ in this equation is an appropriate left preconditioner.
The other routine takes in $y$ and returns $x$, where $x$ is the result of
solving
\begin{equation}
P^T x = y.
\label{eq:pt}
\end{equation}
Here $P^T$ is the transpose of $P$, the preconditioner.
The prototype for both routines is:
\begin{verbatim}
gsl_vector *
user_routine(
  gsl_vector *p_input_vector,
  void *p_user_data
);
\end{verbatim}
The routines take in the vector $y$ as a pointer to a gsl\_vector.
The user also supplies the preconditioner through his or her own data
structure.  The routines take in the pointer to the user's data structure,
appropriately cast.  Typically the user's data structure will supply either
$P$, or, more likely, $P^{-1}$.  Note that the formal solution to
Eq. (\ref{eq:pt}) is
\begin{equation}
x = \left(P^T\right)^{-1} y.
\end{equation}
However,
\begin{equation}
\left( P P^{-1} \right )^T = \left( P^{-1} \right )^T P^T = I,
\end{equation}
where $I$ is the identity matrix.  Hence,
\begin{equation}
\left( P^T \right )^{-1} = \left( P^{-1} \right )^T.
\end{equation}
A logical strategy then is to provide $P^{-1}$ and to write routines that
return $P^{-1} y$ and $\left(P^{-1}\right)^T y$, that is, a matrix vector
multiply and a transpose matrix vector multiply.  For example, if the
matrix $A$ is diagonally dominant, a good choice for $P$ is a simply the
matrix with the diagonal elements of $A$ along the diagonals of $P$ and
the remaining elements of $P$ equal to zero, that is, the element
on row $i$, column $j$ of $P$ is $P_{i,j} = A_{i,i}\delta_{i,j}$, where
$\delta_{i,j}$ is the Kr\"onecker delta.  The elements of $P^{-1}$ then are
$\left(P^{-1}\right)_{i,j} = \delta_{i,j} / A_{i,j}$.

Once the user has constructed the preconditioner solver and preconditioner
transpose solver, he or she passes these in to the sparse matrix solver
via the API routines\\ \\
{\em WnSparseSolve\_\_Mat\_\_updatePreconditionerSolver()}
\\ \\
and
\\ \\
{\em WnSparseSolve\_\_Mat\_\_updatePreconditionerTransposeSolver()}.
\\ \\
The user's data structure is passed into the sparse matrix solver via
the API routine
\\ \\
{\em WnSparseSolve\_\_Mat\_\_updatePreconditionerUserData()}.
\\ \\ 
The wn\_sparse\_solve API documentation and the example codes provided in
the wn\_sparse\_solve distribution demonstrate how to carry out these steps.

\section{Stopping Criteria}

The standard stopping criterion for an iterative solver is when the residual,
defined as
\begin{equation}
{\rm residual} = || Ax - b ||
\label{eq:residual}
\end{equation}
reaches a sufficiently small value.  In Eq. (\ref{eq:residual}), the $||.||$
denotes the two-norm of a vector.  If $v \equiv (v_1, v_2, ..., v_n)$, then
\begin{equation}
||v|| = \sqrt{\sum_{i=1}^n v_i^2}.
\end{equation}
Clearly if the residual is precisely zero, $x$ is the solution to $Ax = b$.

wn\_sparse\_solve provides three convergence methods.  The default method
is ``initial residual'' method.  Here convergence is determined by comparing
the residual relative to the initial residual.  Thus, if $x_{guess}$
is the user-supplied guess vector, then the initial residual is
\begin{equation}
{\rm residual} = || Ax_{guess} - b ||
\label{eq:initial_residual}
\end{equation}
After $i$ iterations, the solution vector is $x_i$, and iteration stops
if
\begin{equation}
|| Ax_{i} - b || \leq \epsilon_{rel}|| Ax_{guess} - b || + \epsilon_{abs}.
\label{eq:initial_residual_convergence}
\end{equation}
Here $\epsilon_{rel}$ is the relative tolerance and $\epsilon_{abs}$ is
the absolute tolerance.

Instead of the ``initial residual'' method, the user can use the ``rhs''
(right-hand-side) method.  Here the convergence criterion is determined
from the condition
\begin{equation}
|| Ax_{i} - b || \leq \epsilon_{rel}|| b || + \epsilon_{abs}.
\label{eq:rhs_convergence}
\end{equation}
The user can switch between these methods via the API routine
\\ \\
{\em WnSparseSolve\_\_Mat\_\_updateConvergenceMethod()}.
\\ \\ 

For certain solvers, the user can also choose to supply his or her own
convergence tester.  The prototype for the user's routine is
\begin{verbatim}
int
user_tester_routine(
  gsl_vector *p_change_vector,
  gsl_vector *p_solution_vector,
  void *p_user_data
);
\end{verbatim}
Here p\_solution\_vector is a pointer to a
gsl\_vector containing the current solution
vector and p\_change\_vector is a pointer to a gsl\_vector containing
the change in the solution vector over the last iteration.  p\_user\_data
is a pointer to a user-defined data structure containing any extra data
the user may wish to use in his or her routine.  The user's routine
must return 0 (false) if the user's convergence criterion is not met and
1 (true) if it is.

Once the user-supplied convergence tester routine is written, the user
provides it to the matrix solver via the API routine
\\ \\
{\em WnSparseSolve\_\_Mat\_\_updateConvergenceTester()}.
\\ \\
The extra data for the convergence tester are set through the API routine
\\ \\
{\em WnSparseSolve\_\_Mat\_\_updateConvergenceTesterUserData()}.
\\ \\
The wn\_sparse\_solve API documentation and an example in the wn\_sparse\_solve
distribution demonstrate how to supply a user-defined convergence tester.
  
\end{document}
