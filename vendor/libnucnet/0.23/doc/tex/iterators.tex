%///////////////////////////////////////////////////////////////////////////////
% <file type="public">
%
% <license>
%   See the README_module.xml file for this module for copyright and license
%   information.
% </license>
%   <description>
%     <abstract>
%       This is the Webnucleo Report for libnucnet iterators.
%     </abstract>
%     <keywords>
%       libnucnet Module, webnucleo report, iterators
%     </keywords>
%   </description>
%
%   <authors>
%     <current>
%       <author userid="mbradle" start_date="2008/07/21" />
%     </current>
%     <previous>
%     </previous>
%   </authors>
%
%   <compatibility>
%     TeX (Web2C 7.4.5) 3.14159 kpathsea version 3.4.5
%   </compatibility>
%
% </file>
%///////////////////////////////////////////////////////////////////////////////
%
% This is a sample LaTeX input file.  (Version of 9 April 1986)
%
% A '%' character causes TeX to ignore all remaining text on the line,
% and is used for comments like this one.

\documentclass{article}    % Specifies the document style.

\usepackage[dvips]{graphicx}
\usepackage{hyperref}

\def\apj{Astrophys. J}

                           % The preamble begins here.
\title{Webnucleo Technical Report: libnucnet Iterators }

\author{Bradley S. Meyer}
%\date{December 12, 1984}   % Deleting this command produces today's date.

\begin{document}           % End of preamble and beginning of text.

\maketitle                 % Produces the title.


This technical report describes how to iterate over species, reactions,
or zones with libnucnet and how to apply functions during the iterations.

\section{Iterations}

Species, reactions, and zones are stored in hashes and lists in libnucnet.
As of version 0.2 of libnucnet, a user loops over them by calling
an iterator and supplying an iterate function to apply during the iteration.
As of version 0.3, it is possible to iterate over reaction elements (reactants
and products) and over optional properties assigned to a zone.

\section{Iterating over Nuclear Species}

To iterate over the species in a collection of species stored
in a Libnucnet\_\_Nuc structure, the user first writes a routine to apply
to each species during the iteration.  The prototype is

\begin{verbatim}
int
my_iterate_function(
  Libnucnet__Species *p_species,
  void *p_my_data
);
\end{verbatim}

The routine may have any appropriate name.  The necessary inputs are:

\begin{itemize}

\item {\bf p\_species:}  A pointer to a species in the collection.

\item {\bf p\_my\_data:}  A pointer to a user-defined data structure containing
extra data to be passed to the user's iterate function.

\end{itemize}

\noindent The routine must return 1 to continue iteration or 0 (zero) to stop.

The user then calls the function to apply with the Libnucnet\_\_Nuc API
routine Libnucnet\_\_Nuc\_\_iterateSpecies() routine, which has the prototype

\begin{verbatim}
void
Libnucnet__Nuc__iterateSpecies(
  Libnucnet__Nuc *self,
  Libnucnet__Species__iterateFunction pf_function,
  void *p_user_data
);
\end{verbatim}

\noindent The necessary inputs are:

\begin{itemize}

\item {\bf self:} A pointer to a Libnucnet\_\_Nuc structure, which contains
the collection of nuclear species.

\item {\bf pf\_function:} The name of the user's function to be applied
during the iteration.
Typically, this needs to be cast as a
Libnucnet\_\_Species\_\_iterateFunction.

\item {\bf p\_my\_data:}  A pointer to a user-define data structure containing
extra data to be passed to the user's iterate function.  If there are
no extra data, this should be NULL.

\end{itemize}

For example, suppose we want to count the number of species with
$Z \geq 10$ in the
Libnucnet\_\_Nuc structure pointed to by p\_my\_nuclei and print out
their name.  We first write an iterate function, which we will
call my\_counter\_and\_printer:

\begin{verbatim}
int
my_counter_and_printer(
  Libnucnet__Species *p_species,
  int *p_count
)
{

  if( !p_species || !p_count ) 
  {
    fprintf( stderr, "Problem with species or user data.\n" );
    return 0;
  }

  if( Libnucnet__Species__getZ( p_species ) >= 10 )
  {
    printf(
      "Species number %d is %s\n",
      *p_count++,
      Libnucnet__Species__getName( p_species )
    );
  }

  return 1;

} 

\end{verbatim}

\noindent Then to apply this routine, the user calls it from his or her program:

\begin{verbatim}
i_count = 0;
Libnucnet__Nuc__iterateSpecies(
  p_my_nuclei,
  (Libnucnet__Species__iterateFunction) my_counter_and_printer,
  &i_count
);
\end{verbatim}

In this example, the code initializes i\_count to zero and then iterates
over the species included in p\_my\_nuclei and applies my\_counter\_and\_printer
to each species.  Note that the species are iterated in the order in which
they were stored [or were previously sorted, if the user previously called
Libnucnet\_\_Nuc\_\_sortSpecies()].
Examples in the libnucnet distribution provide further
details and examples on how to write, apply, compile, and link iterators.

\section{Iterating over Reactions} 

To iterate over reactions in a reaction collection,
the user supplies a routine to apply to a
reaction iteration.  The prototype is

\begin{verbatim}
int
my_reaction_iterate_function(
  Libnucnet__Reaction *p_reaction,
  void *p_my_data
);
\end{verbatim}

\noindent The routine may have any appropriate name.  The necessary inputs are:

\begin{itemize}

\item {\bf p\_reaction:}  A pointer to a reaction.

\item {\bf p\_my\_data:}  A pointer to a user-defined data structure containing
extra data to be passed to the user's iterate function.

\end{itemize}

\noindent
The routine must return 1 to continue iteration or 0 (zero) to stop.

The user then calls the function to apply with the Libnucnet\_\_Reac API
routine Libnucnet\_\_Reac\_\_iterateReactions() routine, which has the prototype

\begin{verbatim}
void
Libnucnet__Reac__iterateReactions(
  Libnucnet__Reac *self,
  Libnucnet__Reaction__iterateFunction pf_function,
  void *p_user_data
);
\end{verbatim}

\noindent
The necessary inputs are:

\begin{itemize}

\item {\bf self:} A pointer to a Libnucnet\_\_Reac structure, which contains
the collection of reactions.

\item {\bf pf\_function:} The name of the user's function to be applied
during the iteration.
Typically, this needs to be cast as a
Libnucnet\_\_Reaction\_\_iterateFunction.

\item {\bf p\_my\_data:}  A pointer to a user-defined data structure containing
extra data to be passed to the user's iterate function.  If there are
no extra data, this should be NULL.

\end{itemize}

Again, examples in the libnucnet distribution provide further demonstration of
the user of reaction iterators.  As of version 0.4, reactions are iterated
in the order in which they are stored internally in the Libnucnet\_\_Reac
structure.  The user does not define this.  To iterate in a different order,
the user supplies a Libnucnet\_\_Reaction\_\_compare\_function and sets it
with Libnucnet\_\_Reac\_\_setReactionCompareFunction().  To restore the
default, the user should call
Libnucnet\_\_Reac\_\_clearReactionCompareFunction().
If the number of reactions is large, the iteration in the default order can be
much faster than one that requires a sorting.  For this reason, the default
should be used where possible.

\section{Iterating over Reaction Elements} 

Reaction elements are reactants or products in a reaction.  As of version
0.3, it is possible to iterate over them.  To do so,
the user supplies a routine to apply to a
reaction element iteration.  The prototype is

\begin{verbatim}
int
my_reaction_element_iterate_function(
  Libnucnet__Reaction__Element *p_element,
  void *p_my_data
);
\end{verbatim}

\noindent The routine may have any appropriate name.  The necessary inputs are:

\begin{itemize}

\item {\bf p\_element:}  A pointer to a reaction element
(a reactant or product).

\item {\bf p\_my\_data:}  A pointer to a user-defined data structure containing
extra data to be passed to the user's iterate function.

\end{itemize}

\noindent
The routine must return 1 to continue iteration or 0 (zero) to stop.

The user then calls the function to apply with the Libnucnet\_\_Reac API
routine Libnucnet\_\_Reaction\_\_iterateReactants() or
Libnucnet\_\_Reaction\_\_iterateProducts() routine, which have the prototypes

\begin{verbatim}
void
Libnucnet__Reaction__iterateReactants(
  Libnucnet__Reaction *self,
  Libnucnet__Reaction__Element__iterateFunction pf_function,
  void *p_user_data
);
\end{verbatim}

\noindent and

\begin{verbatim}
void
Libnucnet__Reaction__iterateProducts(
  Libnucnet__Reaction *self,
  Libnucnet__Reaction__Element__iterateFunction pf_function,
  void *p_user_data
);
\end{verbatim}

\noindent
The necessary inputs for both are:

\begin{itemize}

\item {\bf self:} A pointer to a Libnucnet\_\_Reaction structure, which contains
the collection of reactants and products.

\item {\bf pf\_function:} The name of the user's function to be applied
during the iteration.
Typically, this needs to be cast as a
Libnucnet\_\_Reaction\_\_Element\_\_iterateFunction.

\item {\bf p\_my\_data:}  A pointer to a user-defined data structure containing
extra data to be passed to the user's iterate function.  If there are
no extra data, this should be NULL.

\end{itemize}

Again, examples in the libnucnet distribution provide further demonstration of
the user of reaction element iterators.  Note that the reaction elements
are iterated in the order in which they were stored.  The Libnucnet\_\_Reac
API routine Libnucnet\_\_Reaction\_\_Element\_\_isNuclide() is convenient
for distinguishing between nuclide and non-nuclide reactants or products
while the routine Libnucnet\_\_Reaction\_\_Element\_\_getName() retrieves
the reaction element name.

\section{Iterating over Zones} 

Iterating on zones is analogous to iterating on nuclides, reactions, or
reaction elements.  The iterate function has the prototype

\begin{verbatim}
int
my_iterate_function(
  Libnucnet__Zone *p_zone,
  void *p_my_data
);
\end{verbatim}

\noindent
The routine may have any appropriate name.  The necessary inputs are:

\begin{itemize}

\item {\bf p\_zone:}  A pointer to a zone.

\item {\bf p\_my\_data:}  A pointer to a user-defined data structure containing
extra data to be passed to the user's iterate function.

\end{itemize}

\noindent
The routine must return 1 to continue iteration or 0 (zero) to stop.

To iterate over the zones, the user then calls
Libnucnet\_\_iterateZones(), which has the prototype:

\begin{verbatim}
void
Libnucnet__iterateZones(
  Libnucnet *self,
  (Libnucnet__Zone__iterateFunction) pf_function,
  void *p_user_data
);
\end{verbatim}

\noindent
The necessary inputs are:

\begin{itemize}

\item {\bf self:} A pointer to a Libnucnet structure, which contains
the collection of zones.

\item {\bf pf\_function:} The name of the user's function to be applied
during the iteration.
Typically, this needs to be cast as a
Libnucnet\_\_Zone\_\_iterateFunction.

\item {\bf p\_user\_data:}  A pointer to a user-defined data structure
containing extra data to be passed to the user's iterate function.  If there
are no extra data, this should be NULL.

\end{itemize}

As of version 0.4, zones are iterated
in the order in which they are stored internally in the Libnucnet
structure.  The user does not define this.  To iterate in a different order,
the user supplies a Libnucnet\_\_Zone\_\_compare\_function and sets it
with Libnucnet\_\_setZoneCompareFunction().  To restore the
default, the user should call
Libnucnet\_\_clearZoneCompareFunction().
If the number of zones is large, the iteration in the default order can be
much faster than one that requires a sorting.  For this reason, the default
should be used where possible.

\section{Iterating over Zone Optional Properties} 

As of version 0.3, it is possible to assign optional properties to a zone.
An optional property is stored as a string and is identified by another
string giving its name and up to two optional tags.  It is likely a user
will retrieve the property by an API routine; nevertheless
it is the case that one may
want to iterate over the optional properties of a zone.  This is possible.
The iterate function to be applied during such an iteration has the prototype

\begin{verbatim}
int
my_iterate_function(
  const char *s_name,
  const char *s_tag1,
  const char *s_tag2,
  const char *s_value,
  void *p_user_data
);
\end{verbatim}

\noindent
The routine may have any appropriate name.
The routine must return 1 to continue iteration or 0 (zero) to stop.
The necessary inputs are:

\begin{itemize}

\item {\bf s\_name:}  A string giving the name of the property.

\item {\bf s\_tag1:}  A string giving the first tag of the property.

\item {\bf s\_tag2:}  A string giving the second tag of the property.

\item {\bf s\_value:}  A string giving the value of the property.

\item {\bf p\_user\_data:}  A pointer to a user-defined data structure
containing extra data to be passed to the user's iterate function.

\end{itemize}

\noindent

To iterate over the zone's optional properties, the user then calls
Libnucnet\_\_iterateZones(), which has the prototype:

\begin{verbatim}
void
Libnucnet__Zone__iterateOptionalProperties(
  Libnucnet__Zone *self,
  const char *s_name,
  const char *s_tag1,
  const char *s_tag2,
  Libnucnet__Zone__optional_property_iterate_function pf_function,
  void *p_user_data
);
\end{verbatim}

\noindent
The necessary inputs are:

\begin{itemize}

\item {\bf self:} A pointer to a Libnucnet\_\_Zone.

\item {\bf s\_name:} The property name.  NULL is considered to match on all
property names.  If a name is supplied, the iteration will be over
properties that match that name.

\item {\bf s\_tag1:} The first tag.  NULL is considered to match on all first
tags.  If tag1 is supplied, the iteration will be over properties that match
that tag.

\item {\bf s\_tag2:} The second tag.  NULL is considered to match on all second
tags.  If tag2 is supplied, the iteration will be over properties that match
that tag.

\item {\bf pf\_function:} The name of the user's function to be applied
during the iteration.
Typically, this needs to be cast as a
Libnucnet\_\_Zone\_\_optional\_property\_iterate\_function.

\item {\bf p\_user\_data:}  A pointer to a user-defined data structure
containing extra data to be passed to the user's iterate function.  If there
are no extra data, this should be NULL.

\end{itemize}

As of version 0.6, properties are iterated in the order in which they are
stored internally.  The user does not have control over this ordering.  This
is a change from previous versions in which the iteration was alphabetical
over the properties according to their name or tag.  The new iteration
scheme can be considerably faster for a large number of properties.
Examples in the libnucnet distribution provide further
details.

%\bibliographystyle{siam}
%\bibliography{clemson}

\end{document}
