%///////////////////////////////////////////////////////////////////////////////
% <file
%   repository_path  = "$Source: /pub/cvsprojects/nucleo/web/home/modules/libstatmech/doc/tex/libstatmech.tex,v $"
%   revision         = "$Revision: 1.8 $"
%   date             = "$Date: 2009/08/24 16:29:47 $"
%   tag              = "$Name:  $"
%   template_version = "cp_template.0.12"
% >
%
% <license>
%   See the README_module.xml file for this module for copyright and license
%   information.
% </license>
%   <description>
%     <abstract>
%       This is the Webnucleo Report for the libstatmech module.
%     </abstract>
%     <keywords>
%       libstatmech Module, webnucleo report
%     </keywords>
%   </description>
%
%   <authors>
%     <current>
%       <author userid="tyu" start_date="2009/06/22" />
%       <author userid="mbradle" start_date="2009/08/l8" />
%     </current>
%     <previous>
%     </previous>
%   </authors>
%
%   <compatibility>
%     TeX (Web2C 7.4.5) 3.14159 kpathsea version 3.4.5
%   </compatibility>
%
% </file>
%///////////////////////////////////////////////////////////////////////////////
%
% This is a sample LaTeX input file.  (Version of 9 April 1986)
%
% A '%' character causes TeX to ignore all remaining text on the line,
% and is used for comments like this one.

\documentclass{article}    % Specifies the document style.

\usepackage[dvips]{graphicx}
\usepackage{hyperref}

                           % The preamble begins here.
\title{Webnucleo Technical Report: libstatmech}  % Declares the document's title.

\author{Tianhong Yu, Bradley S. Meyer}

\begin{document}           % End of preamble and beginning of text.

\maketitle                 % Produces the title.


This technical report describes some details of calculations in the
libstatmech module.

\section{Introduction}

libstatmech is a library of C codes for computing the thermodynamics of
fermion and boson gases.  It is built on top of libxml, the GNOME C xml toolkit.
Users can create fermions and bosons and calculate their thermodynamic quantities
numerically with either default integrands (fully relativistic, non-interacting
particles) or user-supplied ones or with user-defined functions.  Users
can also define their own thermodynamics quantities as functions or integrands.
A well-documented API allows users to incorporate libstatmech into their
own codes, and examples in the libstatmech distribution demonstrate the API.

\section{Fermions and Bosons}

Fermions and bosons are stored as Libstatmech\_\_Fermion and
Libstatmech\_\_Boson structures, respectively in libstatmech.  To create a
fermion or boson, the user must supply the particle's name, its rest mass
in MeV, its multiplicity (usually $2J+1$, where $J$ is the particle's spin),
and its charge (in units of the proton's charge).  When a fermion or boson
is created, it automatically has attached to it five thermodynamic quantities,
namely, the number density, the pressure, the energy density (which
includes the rest mass energy density), the internal energy density
(which does not include the rest mass energy density), and the
entropy density.  These quantities are computed in cgs units.  Once a fermion
or boson is created, a user may compute one of the four thermodynamic
quantities by the API routine Libstatmech\_\_Fermion\_\_computeQuantity()
or Libstatmech\_\_Boson\_\_computeQuantity().  These routines will compute
the various quantities by numerical integration of the default integrands,
which are those for fully relativistic, non-interacting particles (see
\S \ref{sec:default} for the definition of these integrals).  The
user may also invert the number density integrand to compute the chemical
potential.  The five standard quantities are identified by the strings
``number density'', ``pressure'', ``energy density'', and ``entropy density'',
which may also be set by the defined parameters S\_NUMBER\_DENSITY,
S\_PRESSURE, S\_ENERGY\_DENSITY, S\_INTERNAL\_ENERGY\_DENSITY,
and S\_ENTROPY\_DENSITY.

\section{Default Integrands}
\label{sec:default}

This section presents the default integrands used in libstatmech.  They
are for non-interacting, fully relativistic fermions and bosons.  The key
parameters are $m$, the particle rest mass, $g$, the particle multiplicity,
$T$, the Temperature, $\hbar$, Planck's constant divided by $2\pi$, $c$,
the speed of light in vacuum, and $k$, Boltzmann's constant.  We use
GSL defined values of these constants (see the GSL documentation) in all
cases.  We choose the cgs system of units.  The other parameters are
$\alpha$ and $\gamma$.  These are defined as follows:
\[
\alpha = \frac{\mu-mc^2}{kT} \equiv \frac{\mu'}{kT},
\]
where $\mu$ is the full chemical potential and $\mu'$ is the chemical potential
less the particle's rest mass energy, and
\[
\gamma = \frac{mc^2}{kT}.
\]

\subsection{Fermions}

In deriving our integrands, we considered fermions and anti-fermions and
assumed them to be in annihilation equilibrium with the photon field.  Thus,
for example, we assumed the reaction $e^+ + e^- \rightleftharpoons \gamma +
\gamma$, where the $\gamma$ represents a photon.  Because the photon has
zero chemical potential, the equilibrium implies
\[
\mu_{e^+} = -\mu_{e^-}.
\]
We may then write in terms of chemical potentials less the rest mass:
\[
\mu'_{e^+} = -\mu'_{e^-} - 2mc^2,
\]
where $m$ is the electron rest mass.  On division by $kT$, this then becomes
\[
\alpha_{e^+} = -\alpha_{e^-} - 2\gamma.
\]
The resulting integrands are the following:

\begin{center}
{\bf Number Density:}
\end{center}
\[
n_{e^+e^-} = \frac {(mc^2)^3g} {2\pi^2(\hbar c)^3\gamma^3}
  \int_0^{\infty} (x+\gamma)\sqrt{x^2+2\gamma x}
  \left[
    \frac{1}{1+{\exp}(x-\alpha)} -
    \frac{1}{1+{\exp}(x+2\gamma+\alpha)}
  \right]
  \, dx
\]
We note that in the special case that the fermion rest mass is zero, the
default integrand for the number density may be integrated directly.
The result is
\[
n_{e^+e^-} = \frac {(kT)^3g} {2\pi^2(\hbar c)^3}
  \left( \frac{\pi^3}{3}\alpha + \frac{\alpha^3}{3} \right )
\]
As
of version 0.5, libstatmech uses this result in default calculations of the
number density and chemical potential when the fermion rest mass is zero.
\begin{center}
{\bf Pressure:}
\end{center}
\[
P_{e^+e^-} = \frac {(mc^2)^4g} {2\pi^2(\hbar c)^3\gamma^4}
  \int_0^{\infty} (x+\gamma)\sqrt{x^2+2\gamma x}
  \left\{
    {\ln}(1+{\exp}[\alpha-x]) +
    {\ln}(1+{\exp}[-x-2\gamma-\alpha])
  \right\}
  \, dx
\]
\begin{center}
{\bf Energy Density:}
\end{center}
\[
\epsilon_{e^+e^-} = \frac {(mc^2)^4g} {2\pi^2(\hbar c)^3\gamma^4}
  \int_0^{\infty} (x+\gamma)^2\sqrt{x^2+2\gamma x}
  \left[
    \frac{1}{1+{\exp}(x-\alpha)} +
    \frac{1}{1+{\exp}(x+2\gamma+\alpha)}
  \right]
  \, dx
\]
\begin{center}
{\bf Internal Energy Density:}
\end{center}
\[
\epsilon_{e^+e^-} = \frac {(mc^2)^4g} {2\pi^2(\hbar c)^3\gamma^4}
  \int_0^{\infty} x (x+\gamma)\sqrt{x^2+2\gamma x}
  \left[
    \frac{1}{1+{\exp}(x-\alpha)} +
    \frac{1}{1+{\exp}(x+2\gamma+\alpha)}
  \right]
  \, dx
\]
\begin{center}
{\bf Entropy Density:}
\end{center}
\[
s_{e^+e^-} = \frac {k(mc^2)^3g} {2\pi^2(\hbar c)^3\gamma^3}
  \int_0^{\infty} {(x+\gamma)\sqrt{x^2+2\gamma x}}
\]
\[
\left[
    \frac{x-\alpha}{1+{\exp}(x-\alpha)} +
    {\ln}[1+{\exp}(\alpha-x)] +
    {\ln}[1+{\exp}(-x-2\gamma-\alpha)] +
    \frac{x+2\gamma+\alpha}{1+{\exp}(x+2\gamma+\alpha)}
\right]
\, dx
\]

\subsection{Bosons}

In the absence of evidence for a conserved boson number,
we neglected the anti-bosons.  The resulting integrands are the following:

\begin{center}
{\bf Number Density:}
\end{center}
\[
n = \frac {(kT)^3g} {2\pi^2(\hbar c)^3}
  \int_0^{\infty} \frac {(x+\gamma)\sqrt{x^2+2\gamma x}} {{\exp}(x-\alpha)-1}
  \, dx
\]
\begin{center}
{\bf Pressure:}
\end{center}
\[
P = -\frac {(kT)^4g} {2\pi^2(\hbar c)^3} \int_0^{\infty}
  (x+\gamma)\sqrt{x^2+2\gamma x} {\ln}[1-{\exp}(\alpha-x)]
  \, dx
\]
\begin{center}
{\bf Energy Density:}
\end{center}
\[
\epsilon = \frac {(kT)^4g} {2\pi^2(\hbar c)^3} \int_0^{\infty}
  \frac {(x+\gamma)^2\sqrt{x^2+2\gamma x}} {{\exp}(x-\alpha)-1}
  \, dx
\]
\begin{center}
{\bf Internal Energy Density:}
\end{center}
\[
\epsilon = \frac {(kT)^4g} {2\pi^2(\hbar c)^3} \int_0^{\infty}
  \frac {x (x+\gamma)\sqrt{x^2+2\gamma x}} {{\exp}(x-\alpha)-1}
  \, dx
\]
\begin{center}
{\bf Entropy Density:}
\end{center}
\[
s = -\frac {(kT)^3g} {2\pi^2(\hbar c)^3} \int_0^{\infty}
  {(x+\gamma)\sqrt{x^2+2\gamma x}}
  \left\{ {\ln}[1-{\exp}(\alpha-x)] + \frac{\alpha-x}{ {\exp}(x-\alpha)-1}
  \right\}
  \, dx
\]

\section{User-Supplied Functions and Integrands for Thermodynamic Quantities}

If a user wishes to use a function or an intergrand other than the default
version to compute a thermodynamic quantity, he or she must supply those.
To supply a function for a thermodynamic quantity of a fermion, the user
writes a routine with the prototype
\begin{verbatim}
double
fermion_function(
  Libstatmech__Fermion *p_fermion, 
  double d_T,
  double d_mukT,
  void *p_user_data
);
\end{verbatim}
Here, {\bf p\_fermion} is a pointer to a libstatmech fermion structure,
{\bf d\_T} is the temperature in K, {\bf d\_mukT} is the $\mu' / kT$,
the chemical potential (less the rest mass, that is, $\mu' = \mu - mc^2$,
where $\mu$ is the full chemical potential) divided by $kT$,
where $k$ is Boltzmann's constant,
and {\bf p\_user\_data} is a pointer to a user-defined structure carrying
extra data into the routine.  The user's routine need not be named
{\bf fermion\_function}.  Analogously, a user-supplied function for a boson
thermodynamic quantity has the prototype
\begin{verbatim}
double
boson_function(
  Libstatmech__Boson *p_boson, 
  double d_T,
  double d_mukT,
  void *p_user_data
);
\end{verbatim}
For both the fermion and boson function, the user's routine must return the
thermodynamic quantity for the input temperature, $\mu/kT$, and other user
data.

A user may also supply an integrand for a thermodynamic quantity.  Here
the respective prototypes are
\begin{verbatim}
double
fermion_integrand(
  Libstatmech__Fermion *p_fermion,
  double d_x,
  double d_T,
  double d_mukT,
  void *p_user_data
);
\end{verbatim}
and
\begin{verbatim}
double
boson_integrand(
  Libstatmech__Boson *p_boson,
  double d_x,
  double d_T,
  double d_mukT,
  void *p_user_data
);
\end{verbatim}
The input parameters are the same as for the functions.  The additional
parameter {\bf d\_x} is the integration variable.

Once the user has written an appropriate function and integrand, he or she
updates them for the fermion or boson.  For example, to update the
pressure for a fermion {\bf p\_fermion} with function
{\bf my\_pressure\_function} and
{\bf my\_pressure\_integrand}, the user calls
\begin{verbatim}
Libstatmech__Fermion__updateQuantity(
  p_fermion,
  S_PRESSURE,
  (Libstatmech__Fermion__Function) my_pressure_function,
  (Libstatmech__Fermion__Integrand) my_pressure_integrand
);
\end{verbatim}
Now when the user calls Libstatmech\_\_Fermion\_\_computeQuantity for
{\bf p\_fermion} and for the pressure, libstatmech will compute the pressure
by first evaluating my\_pressure\_function for the input temperature,
chemical potential, and other data and will add to it the result of
the numerical integration of the integrand my\_pressure\_integrand.
If either the function or integrand is set to NULL, it will not be used
in the calculation of the quantity.  If the integrand is set to
DEFAULT\_INTEGRAND, the integrand will be reset to the default for that
quantity.

The user may also define his or her own quantity (other than one of the
four standard ones).  To do so, the user writes the appropriate function
and integrand and then sets the quantity for the fermion or boson with
the updateQuantity() routine but with the string giving the quantity name
set appropriately.  The quantity is then computed by calling
computeQuantity with that quantity name.  For example, suppose we have
a fermion pointed to by {\bf p\_fermion}.  Now we write a
Libstatmech\_\_Fermion\_\_Function my\_enthalpy\_function and
a Libstatmech\_\_Fermion\_\_Integrand my\_enthalpy\_integrand that
follow the appropriate prototypes.  To compute the enthalpy at a temperature
of $10^3$ K and $\mu'/kT = -23$ with no extra data, we set
the quantity enthalpy and then compute it:
\begin{verbatim}
Libstatmech__Fermion__updateQuantity(
  p_fermion,
  "enthalpy",
  (Libstatmech__Fermion__Function) my_enthalpy_function,
  (Libstatmech__Fermion__Integrand) my_enthalpy_integrand
);

fprintf(
  stdout,
  "The enthalpy density is %g (ergs/cc)\n",
  Libstatmech__Fermion__computeQuantity(
    p_fermion,
    "enthalpy",
    1000.,
    -23.,
    NULL
  ); 
\end{verbatim}

Examples in the libstatmech distribution further demonstrate how to supply
user-defined functions and integrands and how to apply them to thermodynamic
quantities.

\section{Numerical Calculation of the Integrals}

The integration variable for quantity integrands is, in effect, the magnitude
of the particle momentum converted to energy in units of $kT$;
thus, the full range of integration is from zero to $\infty$.  Users
may reset the integral lower limit with the
updateIntegralLowerLimit API routines.  The integrals are
computed with Gnu Scientific Library (GSL) routines.  It is efficient, when the
$\mu'/kT > 0$, to integrate from the lower limit to $\mu'/kT$ and then from
$\mu'/kT$ to $\infty$.  The former integral is done with the GSL routine
{\bf gsl\_integration\_qags} while the latter is done with
{\bf gsl\_integration\_qagiu}.  When $\mu'/kT \leq 0$, the full integration is
done with {\bf gsl\_integration\_qagiu}.

Numerical integration of a quantity continues until the approximation to
the integral satisfies the absolute tolerance $\epsilon_{abs}$ and
relative tolerance $\epsilon_{rel}$.  By default these are both $10^{-8}$
but the user may update them with the
updateQuantityIntegralAccuracy() API routines.  The default
values provide an excellent compromise between accuracy and speed, and
most users should probably not need to change them.  Users who need the routines
to be fast (up to $\sim$ 3 times faster than the default calculations)
and who can sacrifice some accuracy may wish to increase the
tolerance numbers (tolerances of $10^{-2}$ lead to results that are
still typically accurate up to about four decimal places).

\section{Inversion of the Number Density Integral}

It is often the case that one knows the temperature and number density for
a gas of fermions or bosons and seeks the chemical potential.  To do this,
one must invert the number density integral; that is, given $T$ and $n$, the
temperature and number density, one must find the $\mu'/kT$ such that the
integral of the number density agrees with $n$.  libstatmech does this by
finding the root of the function
\begin{equation}
f = n_0 - n(T,\mu'/kT, X),
\end{equation}
where $n_0$ is the given number density, $n$ is the result of the integration
[or, more generally, the function plus integral that gives the number density
as computed from computeQuantity()],
and $X$ represents other parameters to the number density.  The root is
$\mu'/kT$.  To find this root, libstatmech uses the Brent solver in
GSL.  Iteration proceeds until the root achieves a relative tolerance
of $10^{-12}$.  This tolerance is set by the parameter D\_EPS\_ROOT in
Libstatmech.h.  Our experience is that the chosen value works well.

\section{Temperature Derivatives of Thermodynamic Quantities}

Users may compute temperature derivatives of thermodynamic quantities with
the computeTemperatureDerivatives() API routines.
The temperature derivatives are computed with the GSL routine
{\bf gsl\_deriv\_central}.  The step size for the differentiation is
a fraction 0.001 of the temperature at which the derivative is desired.
This fraction may be changed by setting D\_STEP\_FRACTION in Libstatmech.h
to a different value.

\end{document}
